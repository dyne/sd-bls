\documentclass[conference]{IEEEtran}
\usepackage{cite}
\usepackage{url}
\usepackage{amsmath,amssymb,amsfonts}
\usepackage{algorithmic}
\usepackage{graphicx}
\usepackage{textcomp}
\usepackage{xcolor}
\def\BibTeX{{\rm B\kern-.05em{\sc i\kern-.025em b}\kern-.08em
    T\kern-.1667em\lower.7ex\hbox{E}\kern-.125emX}}

\begin{document}

\title{SD-BLS: Privacy Preserving Selective Disclosure of Verifiable Credentials with Unlinkable Threshold Revocation\\
%% {\footnotesize \textsuperscript{*}Note: Sub-titles are not captured in Xplore and should not be used}
\thanks{Dyne.org Foundation}
}

\author{\IEEEauthorblockN{1\textsuperscript{st} Denis Roio}
\IEEEauthorblockA{
% \textit{dept. name of organization (of Aff.)} \\
\textit{Dyne.org Foundation}\\
Amsterdam, The Netherlands \\
jaromil@dyne.org}
\and
\IEEEauthorblockN{2\textsuperscript{nd} Rebecca Selvaggini}
\IEEEauthorblockA{
\textit{dept. of Mathematics} \\
\textit{University of Trento}\\
Trento, Italy
}
\and
\IEEEauthorblockN{3\textsuperscript{rd} Andrea D'Intino}
\IEEEauthorblockA{
% \textit{dept. name of organization (of Aff.)} \\
\textit{Forkbomb B.V.}\\
Copenhagen, Denmark \\
info@forkbomb.eu}
}

\maketitle

\begin{abstract}
It is of critical importance to design digital identity systems that ensure the privacy of citizens as well as protecting them from issuer corruption. We aim to solve this issue and propose a method for selective disclosure and privacy preserving revocation of digital credentials, using the unique homomorphic characteristics of second order Elliptic Curves and Boneh-Lynn-Shacham (BLS) signatures. Our approach ensures that users can selectively reveal credentials signed by a certain issuer, which can be interactively revoked by a quorum of other agreeing issuers without revealing the identity of users. Our goal is to protect users from issuer corruption by requiring collective agreement among multiple revocation issuers.
\end{abstract}

\begin{IEEEkeywords}
Privacy, Selective disclosure, BLS signatures, Digital credentials
\end{IEEEkeywords}
