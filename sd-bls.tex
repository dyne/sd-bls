\documentclass[conference]{IEEEtran}
% \IEEEoverridecommandlockouts
% The preceding line is only needed to identify funding in the first footnote. If that is unneeded, please comment it out.
\usepackage{cite}
\usepackage{amsmath,amssymb,amsfonts}
\usepackage{algorithmic}
\usepackage{graphicx}
\usepackage{textcomp}
\usepackage{xcolor}
\def\BibTeX{{\rm B\kern-.05em{\sc i\kern-.025em b}\kern-.08em
    T\kern-.1667em\lower.7ex\hbox{E}\kern-.125emX}}
\begin{document}

\title{Privacy Preserving Selective Disclosure and Issuer Revocation of Verifiable Credentials*\\
%{\footnotesize \textsuperscript{*}Note: Sub-titles are not captured in Xplore and should not be used}
\thanks{Dyne.org Foundation}
}

\author{\IEEEauthorblockN{1\textsuperscript{st} Denis Roio}
\IEEEauthorblockA{
% \textit{dept. name of organization (of Aff.)} \\
\textit{Dyne.org Foundation}\\
Amsterdam, The Netherlands \\
jaromil@dyne.org}
\and
\IEEEauthorblockN{2\textsuperscript{nd} Andrea D'Intino}
\IEEEauthorblockA{
% \textit{dept. name of organization (of Aff.)} \\
\textit{The Forkbomb Company}\\
Copenhagen, Denmark \\
andrea@forkbomb.eu}
}

\maketitle

\begin{abstract}
We believe that, when developing digital identities systems, it is of critical importance to design algorithms that ensure the privacy of private citizens as well protect them from possible corruption of issuers. This study introduces an innovative method for unlinkable selective disclosure and revocation of digital credentials, utilizing the unique homomorphic characteristics of the Boneh-Lynn-Shacham (BLS) signatures. Our approach ensures that users can selectively reveal only the necessary attributes, while protecting their privacy across multiple presentations. We also want to protect users from issuer corruption, so we apply a threshold for credential issuance and revocation to mandate a collective agreement among multiple issuers, ensuring that actions such as signature validation and revocation are executed only with consensus.
\end{abstract}

\begin{IEEEkeywords}
Privacy, Selective disclosure, BLS signatures, Digital credentials
\end{IEEEkeywords}

\section{Introduction}
Digital identity systems implement credential issuance and presentation mechanisms so that a person can voluntarily disclose his or her own acquired skills, professed attributes, or completed accomplishments. Credentials are signed by issuer authorities and encapsulated within various forms of digital proofs to be held in digital wallets, empowering individuals to reveal only chosen details to designated recipients, to limit data exposure and permit a user-controlled release of information.

Such systems are known as selective disclosure and this article aims at improving their cryptographic implementation to adhere to basic privacy-by-design standards.

\section{Selective Disclosures Today}

Selective disclosures are being used by nation states across the world in their next generation identity systems, for instance EIDAS2.0 in Europe where the European Digital Identity Wallet Architecture and Reference Framework mandates the use of the SD-JWT standard. Unfortunately the only SD-JWT implementation known and found in identity wallet implementations today adopts simple HMAC based cryptography to generate proofs.

In North America the situation seems to be different: the cryptography adopted is based on the BBS+ algorithm and applied to W3C Verifiable Credentials to obtain an higher degree of privacy.

The different choice of syntax in these two approaches is negligible, being Javascript Web Tokens of W3C Verifiable Credentials doesn't changes much from our point of view. But the cryptographic algorithms adopted have great importance for our purpose and they lack the features necessary to face three important threats which render them unsuitable to be used in real world situations.

\section{Main Threats Considered}

\paragraph{Linkability}

Every presentation of an HMAC proof is identical and this makes it possible for an eavesdropper to trace an holder identity by following disclosures. In order to preserve the privacy of a credential holder the proofs disclosed by the wallet should not be traceable across different presentations. This threat appears to be well mitigated by BBS+ through its Zero Knowledge Proof implementation.

\paragraph{Lack of Revocation}

There is no revocation system designed, either for HMAC or BBS+. This may be mitigated by expiration dates, but in some cases they are not enough and interactive revocation is necessary. Since the choice is left open to developers, we will likely face huge privacy breaches with the adoption of revocation lists, as it has happened already during the COVID19 pandemic.

%% - https://github.com/ministero-salute/it-dgc-verificaC19-android/issues/103
%% - https://osf.io/preprints/lawarxiv/yc6xu


\paragraph{Issuer Corruption}

If the choice of interactive revocation is left to a single issuer, one may unilaterally choose to revoke credentials, without being subject to revision or having to seek consensus with a quorum of issuers. This situation leads to censorship and persecution of engaged individuals like journalists or activists living under dictatorial regimes that may arbitrarily revoke their credentials or even ID cards and passports.

\pagebreak

\section*{References}

Please number citations consecutively within brackets \cite{b1}. The
sentence punctuation follows the bracket \cite{b2}. Refer simply to the reference
number, as in \cite{b3}---do not use ``Ref. \cite{b3}'' or ``reference \cite{b3}'' except at
the beginning of a sentence: ``Reference \cite{b3} was the first $\ldots$''

Number footnotes separately in superscripts. Place the actual footnote at
the bottom of the column in which it was cited. Do not put footnotes in the
abstract or reference list. Use letters for table footnotes.

Unless there are six authors or more give all authors' names; do not use
``et al.''. Papers that have not been published, even if they have been
submitted for publication, should be cited as ``unpublished'' \cite{b4}. Papers
that have been accepted for publication should be cited as ``in press'' \cite{b5}.
Capitalize only the first word in a paper title, except for proper nouns and
element symbols.

For papers published in translation journals, please give the English
citation first, followed by the original foreign-language citation \cite{b6}.

\begin{thebibliography}{00}
\bibitem{b1} G. Eason, B. Noble, and I. N. Sneddon, ``On certain integrals of Lipschitz-Hankel type involving products of Bessel functions,'' Phil. Trans. Roy. Soc. London, vol. A247, pp. 529--551, April 1955.
\bibitem{b2} J. Clerk Maxwell, A Treatise on Electricity and Magnetism, 3rd ed., vol. 2. Oxford: Clarendon, 1892, pp.68--73.
\bibitem{b3} I. S. Jacobs and C. P. Bean, ``Fine particles, thin films and exchange anisotropy,'' in Magnetism, vol. III, G. T. Rado and H. Suhl, Eds. New York: Academic, 1963, pp. 271--350.
\bibitem{b4} K. Elissa, ``Title of paper if known,'' unpublished.
\bibitem{b5} R. Nicole, ``Title of paper with only first word capitalized,'' J. Name Stand. Abbrev., in press.
\bibitem{b6} Y. Yorozu, M. Hirano, K. Oka, and Y. Tagawa, ``Electron spectroscopy studies on magneto-optical media and plastic substrate interface,'' IEEE Transl. J. Magn. Japan, vol. 2, pp. 740--741, August 1987 [Digests 9th Annual Conf. Magnetics Japan, p. 301, 1982].
\bibitem{b7} M. Young, The Technical Writer's Handbook. Mill Valley, CA: University Science, 1989.
\end{thebibliography}
\vspace{12pt}
\color{red}
IEEE conference templates contain guidance text for composing and formatting conference papers. Please ensure that all template text is removed from your conference paper prior to submission to the conference. Failure to remove the template text from your paper may result in your paper not being published.

\end{document}
